   In this section, we provide a high-level overview of our proposed approach without delving into the specifics of each component. A more detailed description of each component will be provided in subsequent subsections.

  As it is common in language modeling context, we assume having access to a  large set of text documents. We first append unique separator tokens to mark the beginning and end of each document and then join the resulting documents to create a huge blob of text, $T$. Our approach has two stages, preprocessing/training and serving/testing. The sooner is an offline and computation heavy process while latter is an online and realtime one.
 
    The language model, described here, is designed to predict and return the single most probable token, given the current context. In Section \ref{FurtherDiscussion}, we will explain how to extend this model to output a probability distribution over the entire vocabulary.

  Let's fix $n$ to be the number of tokens in our context. We are given training data T, a large text corpus obtained by concatenating a large set of text documents with unique document boundary markers.
  Our offline stage involves:
\begin{itemize}
\item Constructing a document set $D$, where each document consists of an n-gram followed by its subsequent token in the text corpus $T$.
\item Creating a retrieval index from the documents in $D$.
\item Training a Scoring Model $M$ that takes as input information about the retrieved documents ending with a specific token $t$ and outputs a score, an unbounded real number, representing the likelihood of t being the next token.
\end{itemize}

\noindent Our online stage, given a context $C$ (an $n$-gram), involves:
\begin{enumerate}
\item Constructing a \textit{query} $q$ from $C$.
\item Retrieving a set of documents $R$ from $D$ that match the query $q$ using our retrieval index.
\item For any token $t$ that appeared as the last token for a document in $R$, computing a score $s_t$ by passing the set of documents in $R$ that end with token $t$ to our Scoring Model $M$.
\item Selecting the token with the highest score as the predicted next token.
\end{enumerate}

\subsection{Document Representation}
  In our approach, both queries and documents (i.e., n-grams) are represented using weighted Directed Acyclic Graphs (DAGs).

Let us assume we are given an $n$-gram $T=\langle t_1​,...,t_n\rangle$. We also have a set of triples, $M$, where each triple consists of: (i) a source token sequence $A=\langle a_1,\cdots,a_i$, (ii) a target token sequence $B=\langle b_1,\cdots,b_j\rangle$, and (iii) a real-valued confidence score $f$ between 0 and 1. An entry $\langle A, B, f\rangle$ in $M$ indicates that the token sequence $A$ is considered a synonym of the token sequence $B$ with a confidence score of $f$. It is important to note that these synonyms are not context-aware.
 
 \begin{Example} The following is an example of a synonym triples:
\begin{itemize}
  \item $\langle (\text{New}, \text{York}), (\text{NY}), 1.0\rangle$
  \item $\langle (\text{SF}), (\text{San}, \text{Francisco}), 1.0\rangle$
  \item $\langle (\text{Our}, \text{Family}), (\text{we}), 0.4\rangle$
  \item $\langle (\text{Travelled}), (\text{Drove}), 0.2\rangle$
  \item $\langle (\text{Travelled}), (\text{Flew}), 0.2\rangle$
  \item $\langle (\text{Travelled}), (\text{took a trip}), 0.2\rangle$
\end{itemize}
\end{Example}

We describe the details of building a DAG for an $n$-gram and synonym set $M$, in Algorithm~\ref{AlgBuildingDAG}. 

\begin{algorithm}[t]
\SetAlgoLined
\KwData{an $n$-gram $T=\langle t_1,\cdots,t_n\rangle$ and $T=\{\langle A, B, f\rangle\}$}
\KwResult{DAG $D$}

$M = [\langle p, B, f\rangle: \langle A, B, f\rangle \in T \text{, and } \langle t_p,\cdots,t_{p+|A|-1}\rangle = A]$. \;
$O = $ Create a list from element of $M$ where
\caption{Building DAG from for $n$-gram.}
\label{AlgBuildingDAG}
\end{algorithm}

In this subsection, we will see how a given large blob of text, segmented by special separator tokens, can be transformed into a searchable index. This index will subsequently be used for efficient document retrieval in the serving phase.

  For the sake of explanation, we will use the 1-billion Word Dataset \citep{chelba2013one} as our running example. This dataset is a corpus of monolingual English text used to compare different language models. It is divided into training and test sets, with the training data comprising approximately 1.1 billion tokens, including separator tokens. In this dataset, each document corresponds to a tokenized sentence. Unfortunately, the sentences have been deduped, and information about their original frequency is not available in the training data. As we will discuss in this section, our approach could potentially leverage this property.

  The data for index creation is prepared in several sequential steps. Each step builds upon the previous step's output, enriching the data structure representing the documents with new fields. We use the following data structure to represent our document (details on each field will be provided as we describe the population process)\footnote{We use Free Pascal syntax.}:

\begin{verbatim}
Document = record
  ID: UInt64;
  QualityScore: Single;
  TargetToken: String;
  Context: array of String;
end;
\end{verbatim}

In the first step, we go through the tokens, $Token$ in $T$, and set  \textit{TargetToken} to be \tt{Token}, and \tt{ID} to be a unique value. The \tt{Context} is set to the tokens from the last “Start of the document” token, inclusively, until the current token, exclusively. The field \tt{QualityScore} can be used to indicate the quality of the original text document. For example, in the case of web documents, we can use Page Rank~\citep{brin1998anatomy}. For our running example, given a sentence,  the \tt{Context} for a \tt{token} is the list of the tokens that are before it, including \tt{\textbackslash S}. Also we set the \tt{QualityScore} field to 1. This means we believe the quality/validity of all our sentences are the same. 

In the next step, we group documents by their \textit{TargetToken} and \textit{Context} fields. For each group of documents with identical \textit{TargetToken} and \textit{Context}, we create a new document with a unique \textit{ID}. The \textit{QualityScore} of this new document is calculated by summing the \textit{QualityScores} of the original documents in the group. While other aggregation functions (such as max) could be considered, we utilize the sum operator in this paper.

In Subsection \ref{EnrichingDoc}, we will present additional enhancements to the \textit{Document} data structure.




\subsection{Index Creation}
  We create two sets of indices: a Retrieval Index and a Forward Index. The retrieval index is an extension of the inverted index \cite{InvertedIndex} used in document retrieval systems. The Forward Index, which is a map from document ID to a structure containing all the necessary information for estimating the probabilities of the next token. 

  \subsubsection{Retrieval Index}
  In the context of language modeling, the traditional approach to building a retrieval index involves creating a map from all possible n-grams to their frequencies. While Suffix Arrays (\cite{stehouwer2010using}, \cite{kennington2012suffix}) offer a space-efficient solution to this problem, they may not be optimal for handling the term level synonyms.

  This paper proposes an alternative approach that addresses the memory limitations of traditional n-gram models. As we will demonstrate in Subsections \ref{SynonymsInDocument} and \ref{SynonymsInQuery}, our approach allows for the expansion of both queries and documents with term-level as well as phrase-level synonyms, an aspect that is computationally challenging for Suffix Array-based methods.

  A key principle underlying our index construction is the retrieval of a superset of matching n-grams rather than the exact set. This approach, while potentially requiring the filtering out of irrelevant documents during the serving phase, significantly reduces memory requirements. As we will demonstrate in this section, the proportion of irrelevant documents within the retrieved set is typically minimal. The process of filtering these irrelevant documents will be described in Section~\ref{Serving}.

  We represent each document as a Weighted Directed Acyclic Graph (WDAG), where nodes represent tokens and edges represent the sequential relationship between tokens. Each edge is associated with a weight, a real value within the range (0, 1]. A source node is added to connect to the node corresponding to the first token of each document, and a sink node is added to connect to the node corresponding to the last token of each document.

  Our retrieval index is effectively a map from pairs of (dist, token), where $dist$ is a non-negative integer and $token \in Vocab$, to Document IDs. To create our retrieval index, we process each node in the DAG for a given document. For each node:
\begin{enumerate}
  \item Determine the set of all possible distances (Dist) to the sink node within the DAG.
  \item Retrieve the corresponding token (t) for the node.
  \item For each distance $d \in Dist$:
    \begin{itemize}
      \item Add the document's ID as a value for the key (d, t) in the retrieval index.
    \end{itemize}
\end{enumerate}

As we will demonstrate in Section~\ref{DocumentSynonyms}, this approach to constructing the retrieval index enables the incorporation of Term/Phrase-level enrichment into the documents. Furthermore, it is important to note that our retrieval index will have $O(N * |vocab|)$ keys.
Figure \ref{DocumentSimplePreProcessingFig} visualizes how we create a retrieval index from the input documents.



  
